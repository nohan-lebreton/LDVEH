\documentclass[12pt]{article}

\usepackage[utf8]{inputenc}%encodage des caractères
\usepackage[T1]{fontenc}%encodage de la police
\usepackage[french]{babel}%langue française
\usepackage{graphicx}
\usepackage{amsmath}

\begin{document}

\begin{titlepage}

\title{Rapport du projet Conception logicielle avancée}
\author{Dorlodot Des Essarts, Duclos, Guillot, Lebreton, Longuet}
\date{04 avril 2022}
\maketitle
\begin{figure}[htp]
  
  \centering
    \includegraphics[width=0.5\textwidth]{Logo_unicaen}
\end{figure}
\vfill
\vfill
\begin{flushleft}
Encadrants :\\
M. Grégory Bonnet\\
M. Meddouri Nida\\
\bigskip
Année : 2021-2022\\
\end{flushleft}

\hfill
\thispagestyle{empty}
\setcounter{page}{0}
\end{titlepage}




\newpage

\tableofcontents

\newpage
	
\section{Introduction}
\subsection{Description du sujet de projet}
Énoncé : Les livres dont vous êtes le héros (ou LDVEH) sont des jeux de rôles en solitaire dont la
narration est décomposées en paragraphes, dispersés dans le livre. Des liens, en fonction des choix
du lecteurs, permettent d’aller d’un paragraphe à l’autre.\medskip
\\
Dans un premier temps, il s’agira de développer un éditeur de texte qui maintienne le graphe du jeu 
et qui puisse réorganiser aléatoirement le livre. L’affichage de la ou les solutions (paragraphe devant être traversés pour
gagner) devront être calculées ainsi que la difficulté du jeu (proportions de solutions au regard des
chemins possibles).\medskip
\\
Dans un second temps, il s’agira d’ajouter un système de rencontres et de combat, ainsi que la gestion d’objets ou d’indices qui peuvent être nécessaire pour gagner. \medskip
\\
Enfin, un mode lecture permettant de jouer au LDVEH pourra être implanté.
\subsection{Présentation du plan du rapport}
\textbf{A faire}

Blabla
\section{Objectif du projet}
\subsection{Interprétation du sujet de projet}
\textbf{A faire}
\subsection{Description de travaux existants sur le même sujet	}
\noindent
\textbf{A développer}

\noindent
http://litteraction.fr/presentation/livre-dont-on-est-le-heros/logiciels \medskip
\\
https://projectaon.org/staff/christian/gamebook.js/
\section{Fonctionnalités implémentées}
\subsection{Description des fonctionnalités}
\textbf{A finir}

\noindent
-Textes et règles de l'histoire personalisables 
-Personaliser l'arbre du déroulement de l'histoire (nombre de branches, de noeuds...)
-Implémentation de l'inventaire du joueur
-Un mode lecture pour l'histoire créée
-(Ajout de son à l'histoire)
-Interface graphique afin de visualiser le programme sans console
-Création du graphique représentatif de l'histoire
\subsection{Organisation du projet}
\textbf{A faire}
\newpage
\section{\'Elements techniques}
\subsection{Description des paquetages non standards utilisés}
\textbf{A faire}

Ceci est un rapport rédigé en LaTeX\footnote{1. langage que nous apprenons aujourd'hui.}. La première sous-section, c'est-àdire la sous-section 1.1 se situe en page 2 (numéro de page calculé automatiquement). Ci-dessous se trouve une liste numérotée d'items :
\begin{enumerate}
  \item bla
  \item ble
  \item bli
  \item blo
  \item blu
\end{enumerate}
\subsection{Description des algorithmes}
\textbf{A faire}

 \begin{table}[htp]
	\centering
		\begin{tabular}{|c|l|r|}
  		\hline
  		texte centré & texte à gauche & texte à droite \\
  		\hline
  		a & b & c \\
  		\hline
  		d & e & f \\
  		\hline
		\end{tabular}
	\caption{Caption below table.}
\end{table}

Le tableau 1 est nommé.
\section{Architecture du projet}
\subsection{Diagrammes des modules et des classes}
\textbf{A faire}

\textbf{Un arbre, la racine étant le point reliant tous les modules, puis pour chaque module, des branches pour chaque classe, puis des branches pour les méthodes.}

Le nombre $\pi$ vaut environ $3.14$, ou encore $\frac{22}{7}$ ou $\frac{\frac{44}{2}}{\sqrt{7^2}}$ à un gros $\epsilon$ près.
\newline 

Une équation non numérotée :
$$ a^2 + b^2 = c^2$$
\newline 

Une équation numérotée :
\begin{equation}
a^2 + b^2 = c^2
\end{equation}

On peut citer l'équation précédente comme étant l'équation 1.
\newline

Avec "align", on peut citer chaque ligne d'une équation. Par exemple,
\newline
l'équation 2 et l'équation 3.
\begin{align}
f(x) &=\qquad x^2 + 8x + 16 \\
	 &=\qquad (x+4)^2 
\end{align}
\subsection{Chaînes de traitements}
\textbf{A faire}

\textbf{Déroulement chronologique du programme, boite par boite}
\section{Expérimentations et usages}
\subsection{Cas d'utilisation}
\textbf{A faire}

\textbf{Mode d'emploi sur un cas d'exemple}
Je cite la première référence [1]. Je peux aussi citer les 3 d'un seul coup
[1, 2, 3].
\begin{thebibliography}{1}
\bibitem{notes} John W. Dower {\em Readings compiled for History 21.479.} 1991.
\bibitem{impj} The Japan Reader {\em Imperial Japan 1800-1945} 1973: Random House, N.Y.
\bibitem{norman} E. H. Norman {\em Japan's emergence as a modern state} 1940: International Secretariat, Institute of Pacific Relations.
\bibitem{fo} Bob Tadashi Wakabayashi {\em Anti-Foreignism and Western Learning in Early-Modern Japan} 1986: Harvard University Press.
\end{thebibliography}
\subsection{Résultats quantifiables}
\textbf{A faire}
\textbf{Résultats obtenus pour une histoire ni trop simple ni trop compliquée (temps d'execution pour afficher le graph, temps de compilation)}
\section{Conclusion}
\textbf{A faire}

\textbf{Résumé de ce que nous a permis de faire et d'apprendre avec le projet}
\end{document} 